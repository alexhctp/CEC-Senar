\documentclass[a4paper, 12pt]{article}

% Pacotes básicos para acentuação, matemática e layout
\usepackage[utf8]{inputenc}
\usepackage[T1]{fontenc}
\usepackage{amsmath}
\usepackage{geometry}

% Pacotes para tabelas aprimoradas
\usepackage{longtable} % Para tabelas que podem quebrar entre páginas
\usepackage{array}     % Para formatação avançada de colunas
\usepackage{booktabs}  % Para linhas de tabela com visual profissional

% Configuração da página com orientação paisagem
\geometry{a3paper, margin=1in, landscape}

\begin{document}

% Título do documento
\begin{center}
    \huge\bfseries Folha de Fórmulas - Planejamento Empreendimento Cafeeiro - Prof. Esaú
\end{center}

\vspace{0.1cm}

% Início da tabela longa
% As colunas são definidas com larguras relativas ao tamanho do texto
\begin{longtable}{@{}p{0.3\textwidth} p{0.3\textwidth} p{0.4\textwidth}@{}}

% Cabeçalho da tabela, que se repete em novas páginas
\toprule
\textbf{A Fórmula} & \textbf{O Objetivo} & \textbf{Descrição dos Itens} \\
\midrule
\endhead

% Rodapé da tabela
\bottomrule
\endfoot

% --- Seção 1: Densidade de Plantio ---
\multicolumn{3}{@{}l}{\textbf{Cálculo de Densidade de Plantio}} \\
\cmidrule(r){1-3}
$$ \text{Nº de Plantas} = \frac{\text{Área Total}}{(\text{Esp. Linhas} \times \text{Esp. Plantas})} $$ &
Calcular o número total de plantas que cabem em uma determinada área. &
\textbf{Nº de Plantas:} Quantidade total de plantas no talhão. \newline
\textbf{Área Total:} Superfície total do terreno (em m²). \newline
\textbf{Esp. Linhas:} Espaçamento entre as linhas de plantio (em m). \newline
\textbf{Esp. Plantas:} Espaçamento entre as plantas na mesma linha (em m). \\
\midrule

% --- Seção 2: Metros Lineares ---
\multicolumn{3}{@{}l}{\textbf{Cálculo de Metros Lineares}} \\
\cmidrule(r){1-3}
$$ \text{Metros Lineares} = \frac{\text{Área Total}}{\text{Espaçamento entre Linhas}} $$ &
Calcular o comprimento total de todas as linhas de plantio. É a base para dimensionar a irrigação. &
\textbf{Metros Lineares:} Comprimento total das fileiras (em m). \newline
\textbf{Área Total:} Superfície total do terreno (em m²). \newline
\textbf{Espaçamento entre Linhas:} Distância entre as fileiras (em m). \\
\midrule

% --- Seção 3: Fita de Gotejamento ---
\multicolumn{3}{@{}l}{\textbf{Cálculo da Fita de Gotejamento}} \\
\cmidrule(r){1-3}
$$ \text{Comp. da Fita} = \text{Metros Lineares} \times \left(1 + \frac{\% \text{ Acréscimo}}{100}\right) $$ &
Dimensionar a quantidade total de fita de gotejamento necessária, incluindo uma margem para perdas e manobras. &
\textbf{Comp. da Fita:} Comprimento total da fita de gotejamento a ser comprada (em m). \newline
\textbf{Metros Lineares:} Comprimento total das fileiras (em m). \newline
\textbf{\% Acréscimo:} Percentual extra para perdas e conexões (geralmente 3\% a 5\%). \\
\midrule

% --- Seção 4: Emissores ---
\multicolumn{3}{@{}l}{\textbf{Cálculo de Emissores (Gotejadores)}} \\
\cmidrule(r){1-3}
$$ \text{Nº de Emissores} = \frac{\text{Metros Lineares}}{\text{Espaçamento entre Emissores}} $$ &
Determinar a quantidade total de pontos de gotejamento (emissores) em todo o sistema. &
\textbf{Nº de Emissores:} Quantidade total de gotejadores. \newline
\textbf{Metros Lineares:} Comprimento total das fileiras de plantio (em m). \newline
\textbf{Espaçamento entre Emissores:} Distância entre os gotejadores na fita (em m). \\
\midrule

% --- Seção 5: Vazão Total ---
\multicolumn{3}{@{}l}{\textbf{Cálculo de Vazão Total}} \\
\cmidrule(r){1-3}
$$ \text{Vazão Total} = \text{Nº de Emissores} \times \text{Vazão por Emissor} $$ &
Calcular a demanda total de água do sistema de irrigação por hora, essencial para dimensionar a bomba. &
\textbf{Vazão Total:} Volume total de água necessário por hora (em L/h). \newline
\textbf{Nº de Emissores:} Quantidade total de gotejadores. \newline
\textbf{Vazão por Emissor:} Vazão individual de cada gotejador (em L/h). \\
\midrule

% --- Seção 6: Insumos por Planta ---
\multicolumn{3}{@{}l}{\textbf{Cálculo de Insumos por Planta}} \\
\cmidrule(r){1-3}
$$ \text{g/planta} = \frac{(\text{Dose Total} \times \text{Fator de Conversão})}{\text{Nº de Plantas}} $$ &
Calcular a quantidade exata de insumo (fertilizante, corretivo) a ser aplicada em cada planta individualmente. &
\textbf{g/planta:} Gramas do insumo a serem aplicadas por planta. \newline
\textbf{Dose Total:} Quantidade total do insumo recomendada para a área (em Kg ou T). \newline
\textbf{Fator de Conversão:} Usar \textbf{1.000} para converter Kg para g; \textbf{1.000.000} para converter T para g. \newline
\textbf{Nº de Plantas:} Quantidade total de plantas na área. \\
\midrule

% --- Seção 7: Insumos por Metro Linear ---
\multicolumn{3}{@{}l}{\textbf{Cálculo de Insumos por Metro Linear}} \\
\cmidrule(r){1-3}
$$ \text{g/metro} = \frac{(\text{Dose Total} \times \text{Fator de Conversão})}{\text{Metros Lineares}} $$ &
Calcular a quantidade de insumo a ser aplicada por metro de linha de plantio, comum em fertirrigação ou aplicação em sulco. &
\textbf{g/metro:} Gramas do insumo por metro linear. \newline
\textbf{Dose Total:} Quantidade total do insumo (em Kg ou T). \newline
\textbf{Fator de Conversão:} 1.000 para Kg; 1.000.000 para T. \newline
\textbf{Metros Lineares:} Comprimento total das fileiras (em m). \\

\end{longtable}

\end{document}
